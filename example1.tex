\documentclass{beamer} %Telling Latex this is a beamer (slides) document
\usetheme{Singapore} %Choice of there. There are many, you pick yours. 
\useoutertheme[subsection=false]{smoothbars} %horizontal "counting circles" (known as navigation bar). Try running with %and without this definition and see what happens. You will understand. 
\usepackage{hyperref} %package I use in Latex for references

%Customization (non-necessary)
\setbeamertemplate{itemize items}[ball] % if you want a ball for your items	
\setbeamertemplate{itemize subitem}[triangle] % if you want a circle for your subitems

%Creating a title slide
\title{Introduction: Advanced Quantitative Methods}
\author{Nat\'{a}lia S. Bueno
\\natalia.bueno@yale.edu
\\Department of Political Science
\\Yale University}
\date{August 30, 2013}

\begin{document} % begin your document

\frame{\titlepage} %for your first slide, have your title page

\section{Objectives}

\subsection{} %You need this because of the "navigation bar" (counting circles on top)
\frame %Here is how you start each slide
{ %beginning the slides
\frametitle{Objectives} %title for a slide, not for section
%%%%%%%%%%%%%%%%%%%%%% Write like you would in a Latex document
\begin{itemize}
\item Welcome!
\item Logistics.
\item What to expect from PLSC 504.
\item R training.
\end{itemize}
} %End the slide

\section{Logistics \& Expectations}

\subsection{}
\frame
{
\frametitle{Logistics}
\begin{itemize}
\pause \item The basics: %Pauses are a good thing in presentation. You can pause before items, %you can pause a word in the middle of a sentence. 
\begin{itemize}
\item Contact: e-mail and class forum on classes*V2.
\item \href{https://www.dropbox.com/sh/8gos1knztfhdgty/IzxBeqKvUW}{Dropbox folder} with materials (lecture and section slides, homework, solutions, readings, and R and \LaTeX{} materials).
\item classesV2 for grades and uploading assignments.
\end{itemize}
\pause \item Office hours. M: 4:00-5:00 and T: 3:00-4:00, place RZK 220.   
\end{itemize}
}


\subsection{}
\frame
{
\frametitle{Sections}
\begin{itemize}
\item Section time. Probably on Wednesdays, place TBD.
\item Q \& A, lecture review, and new content.
\item I encourage you to e-mail me questions before section so you can get better answers. 
\item Evaluation/Attendance.
\end{itemize}
}

\subsection{}
\frame{
\frametitle{Homework \& Final Exam}
\begin{itemize}
\pause \item Homework.
\begin{itemize}
\item Graded on effort (your chance to make mistakes)
\item Individual assignment, but you can work in groups (ideally after your own attempts at solving it).
\item If you have trouble finding a study group, please let me know.
\end{itemize}
\pause \item Final exam.
\begin{itemize}
\item Good pratice for the qualifying exam.
\end{itemize}
\end{itemize}
}

\subsection{}
\frame{
  \frametitle{Replication Paper}
  \begin{itemize}
  \pause \item Learn by doing (and by mimicking).
  \pause \item Jobs and replication papers.
  \pause \item Get data and replication files quickly.
\begin{itemize} 
\item 18 of 120 political science journals have replication policies
\item Only 16\% of surveyed individuals (students + PolMeth) were able to get complete dataset and replication files. 21\% got a limited dataset and a complete code. The plurarity (24\%) got a limited dataset but no code.
\item Replication tips: \href{http://politicalsciencereplication.wordpress.com/}{Replication blog}
\end{itemize}
  \pause \item Deadlines to keep you on track.
  \pause \item Work in small groups (2-3 people): again, let me know if you have trouble finding partners.
  \end{itemize}
}

%End your document
\end{document}
